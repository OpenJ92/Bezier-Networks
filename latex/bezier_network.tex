\documentclass{article}
\usepackage{amsmath}

\title{Bezier Neural Networks}
\author{Jacob Martin Vartuli-Schonberg}
\date{\today}

\begin{document}


\maketitle
\tableofcontents
\newpage

\section{Bezier Curves: Recursive Construction}

define the first and last row of this.

We will first define and inspect the core object of study in our paper, the Bezier Curve. To my mind,
the Bezier Curve is a function mapping from \( R^{m x n} \to R^{m x t} \) parameterized be a real number \(t \in [0,1]\).
In effect, we're mapping an m by n matrix \(A^{m x n}\) into a function of a single parameter \(t\). One might write this
relation in the function \(B\) as

\begin{equation} \label{1} B^{n-1}(A) = b(t|A) \end{equation}

where \(B\) is an opperator on a matrix and \(b\) is a function \( R^{m x n} \to R^{1} \to R^{m} \). With this abstraction, 
let us now define explicitly. We define an opperator \(B | R^{m x m x t} \to R^{m x {n-1 x t}}\) on a matrix \(A^{m x n}\) as 

\begin{equation} B(A^1) = A^2(t) \end{equation}

Particularly, this opperation is a convolution of the column space of \(A_1\) and comes in the form of

\begin{subequations}
  \begin{equation}
    B(A^1)_i = A^1_i + t(A^1_{:,i-1} - A^1_{:,i})
  \end{equation}
  \begin{equation}
    B(A^1)_i = (1 - t)A^1_{:,i} + A^1_{:,i-1}
  \end{equation}
\end{subequations}

As you can see, the resulting matrix upon one operation results in a matrix \(A^{nxm-1|t} \). Suppose we apply this operation
\(l\) times upon that same matrix where \(l < n\). We would then arrive at the matrix 

\begin{equation}
  B^l(A)_{:, i} = (1-t)B^{l-1}(A)_{:,i} + B^{l-1}(A)_{:,i-1}
\end{equation}
if \(l = n\), we arrive at a vector valued function which is named a bezier curve.

\begin{equation}
  b(t|A) = B^m(A)_{:, i} = (1-t)B^{l-1}(A)_{:,i} + B^{l-1}(A)_{:,i-1}
\end{equation}

We will soon come to see that a treatment of such a curve will enable us to construct a myriad or family of Neural Networks
given a set of control points or as given above, a matrix \(A\)

\begin{equation}
	\frac{\delta b(t|A)}{\delta A} \in R^{nxnxm|t}
\end{equation}
\begin{equation}
	\frac{\delta b(t|A)}{\delta t} \in R^{n|t}
\end{equation}
where equation 6 will be of the utmost imporatnce in the comming sections.

\section{On the Resolution of Convolution Kernal Parameters}
Consider the selection of the parameters of the Kernel of a Convolutional Network. As far as I can see, there is no consideration outside
a desire to resolve such Kernels in the static situation. In this section, we will take into consideration these formula in detail and seek 
to apply them to the dynamic case.

\begin{equation}
	a^2 + b^2 = c^2
\end{equation}

Ultimately, these forms are of the linear flavor. What are some questions that one should consider? Consider the number of solutions for moving
from one layer to the next. Consider a collection of policies which one might employ for choosing from such a solution set and how a choice effects
choices later in the process. What are the choices that are invarient in this process? etc ...

\section{On the Synthesis of Bezier and Kernal Parameters}
With a firm understanding of the Bezier construction and f the parameters of the Kernel, we now seek to synthesize such concepts so as to introduce
a new means to hyperparameterize Neural Networks for training.

\section{Testing - Experiment}

\end{document}
