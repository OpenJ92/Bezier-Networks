\documentclass{article}
\usepackage{amsmath}

\title{Bezier Neural Networks}
\author{Jacob Martin Vartuli-Schonberg}
\date{\today}

\begin{document}

\maketitle
\tableofcontents
\newpage

\section{Bezier Curves: Recursive Construction}

define the first and last row of this.

We will first define and inspect the core object of study in our paper, the Bezier Curve. To my mind,
the Bezier Curve is a function mapping from \( R^{m x n} \to R^{m x t} \) parameterized be a real number \(t \in [0,1]\).
In effect, we're mapping an m by n matrix \(A^{m x n}\) into a function of a single parameter \(t\). One might write this
relation in the function \(B\) as
\begin{equation} \label{1} B^{n-1}(A) = b(t|A) \end{equation}
where \(B\) is an opperator on a matrix and \(b\) is a function \( R^{m x n} \to R^{1} \to R^{m} \). With this abstraction, 
let us now define explicitly. We define an opperator \(B | R^{m x m x t} \to R^{m x {n-1 x t}}\) on a matrix \(A^{m x n}\) as 
\begin{equation} B(A^1) = A^2(t) \end{equation}
Particularly, this opperation is a convolution of the column space of \(A_1\) and comes in the form of

\begin{subequations}
  \begin{equation}
    B(A^1)_i = A^1_i + t(A^1_{:,i-1} - A^1_{:,i})
  \end{equation}
  \begin{equation}
    B(A^1)_i = (1 - t)A^1_{:,i} + A^1_{:,i-1}
  \end{equation}
\end{subequations}

As you can see, the resulting matrix upon one operation results in a matrix \(A^{nxm-1|t} \). Suppose we apply this operation
\(l\) times upon that same matrix where \(l < n\). We would then arrive at the matrix 
\begin{equation}
  B^l(A)_{:, i} = (1-t)B^{l-1}(A)_{:,i} + B^{l-1}(A)_{:,i-1}
\end{equation}
if \(l = n\), we arrive at a vector valued function which is named a bezier curve.
\begin{equation}
  b(t|A) = B^m(A)_{:, i} = (1-t)B^{l-1}(A)_{:,i} + B^{l-1}(A)_{:,i-1}
\end{equation}

\section{On the Resolution of Convolution Kernal Parameters}
\section{On the Synthesis of Bezier and Kernal Parameters}

\end{document}
